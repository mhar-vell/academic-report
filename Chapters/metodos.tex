\chapter{Materiais e Métodos}
\label{chap:mat}
asdfasdfsdf

\section{Metodologia}
\label{sec:met}
adadfasf

%--------- NEW SECTION ----------------------
\section{Interface do Usuário}
\label{sec:ui}
asdfadsfsdfs

%--------- NEW SECTION ----------------------
\section{Simulação do sistema}
\label{sec:sim}
asdfadsfsdfs


