\chapter{Fundamentação Teórica}
\label{chap:fundteor}

\begin{flushright}

   \begin{list}{}{
      \setlength{\leftmargin}{4.5cm}
      \setlength{\rightmargin}{0cm}
      \setlength{\labelwidth}{0pt}
      \setlength{\labelsep}{\leftmargin}}
      \item Quanto maior for a rapidez de transformação de uma
      sociedade, mais temporárias são as necessidades
      individuais. Essas flutuaçõess tornam ainda mais acelerado
      o senso de turbilh da sociedade.

      \begin{list}{}{
      \setlength{\leftmargin}{0cm}
      \setlength{\rightmargin}{0cm}
      \setlength{\labelwidth}{0pt}
      \setlength{\labelsep}{\leftmargin}}
      \item (Alvin Toffler)
      \end{list}
   \end{list}
\end{flushright}

\begin{flushright}
  Quanto maior for a rapidez de transformação de uma \\
  sociedade, mais temporárias são as necessidades \\
  individuais. Essas flutuações tornam ainda mais \\
  acelerado o senso de turbilhão da sociedade. \\
  \ \\
  (Alvin Toffler)
\end{flushright}

%--------- NEW SECTION ----------------------
\section{Estudo do estado da arte}
\label{sec:sota}
flkjasdlkfjasdlkfj

%---------------picture------------------------------------
\begin{figure} [h!]												 
	\centering													 
	\includegraphics[width=0.6\textwidth]{./lq}				 
	\caption{Insufficient data.}		
	\label{img:ihuma}												 
\end{figure}													 
%----------------------------------------------------------

%--------- NEW SECTION ----------------------
\section{Assunto 1}
\label{sec:ass1}
flkjasdlkfjasdlkfjs



%---------------picture------------------------------------
% \begin{figure}
%     \centering
%     \subfigure[Figure A]{\label{fig:a}\includegraphics[width=60mm]{./lq}}
%     \subfigure[Figure B]{\label{fig:b}\includegraphics[width=60mm]{./lq}}
%     \subfigure[Figure C]{\label{fig:c}\includegraphics[width=\textwidth]{./lq}}
%     \caption{Three simple graphs}
%     \label{fig:three graphs}
% \end{figure}
%----------------------------------------------------------

\begin{figure}
    \centering
    \begin{subfigure}[b]{0.3\textwidth}
        \centering
        \includegraphics[width=\textwidth]{./lq}
        \caption{$y=x$}
        \label{fig:y equals x}
    \end{subfigure}
    \hfill
    \begin{subfigure}[b]{0.3\textwidth}
        \centering
        \includegraphics[width=\textwidth]{./lq}
        \caption{$y=3sinx$}
        \label{fig:three sin x}
    \end{subfigure}
    \hfill
    \begin{subfigure}[b]{0.3\textwidth}
        \centering
        \includegraphics[width=\textwidth]{./lq}
        \caption{$y=5/x$}
        \label{fig:five over x}
    \end{subfigure}
       \caption{Three simple graphs}
       \label{fig:three graphs}
\end{figure}


%--------- NEW SECTION ----------------------
\section{Assunto 2}
\label{sec:ass2}
flkjasdlkfjasdlkfjs

\begin{table}[h]
    \begin{subtable}[h]{0.45\textwidth}
        \centering
        \begin{tabular}{l | l | l}
        Day & Max Temp & Min Temp \\
        \hline \hline
        Mon & 20 & 13\\
        Tue & 22 & 14\\
        Wed & 23 & 12\\
        Thurs & 25 & 13\\
        Fri & 18 & 7\\
        Sat & 15 & 13\\
        Sun & 20 & 13
       \end{tabular}
       \caption{First Week}
       \label{tab:week1}
    \end{subtable}
    \hfill
    \begin{subtable}[h]{0.45\textwidth}
        \centering
        \begin{tabular}{l | l | l}
        Day & Max Temp & Min Temp \\
        \hline \hline
        Mon & 17 & 11\\
        Tue & 16 & 10\\
        Wed & 14 & 8\\
        Thurs & 12 & 5\\
        Fri & 15 & 7\\
        Sat & 16 & 12\\
        Sun & 15 & 9
        \end{tabular}
        \caption{Second Week}
        \label{tab:week2}
     \end{subtable}
     \caption{Max and min temps recorded in the first two weeks of July}
     \label{tab:temps}
\end{table}